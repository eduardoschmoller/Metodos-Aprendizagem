\documentclass[12pt]{article}

\usepackage{sbc-template}

\usepackage{graphicx,url}
\usepackage[brazil]{babel}
\usepackage[utf8]{inputenc}
\usepackage[hidelinks]{hyperref}


\sloppy

\title{Smartphone-Based Recognition of Human Activities and Postural Transitions}

\author{Bruno M. Dobrovolski\inst{1}, Eduardo A. Schmoller\inst{1}}

\address{Departamento Acadêmico de Informática -- Universidade Tecnológica Federal do Paraná\\
	Pato Branco -- PR -- Brasil	
	\email{\{brunod,schmoller\}@alunos.utfpr.edu.br}
}

\begin{document} 

\maketitle

%\begin{abstract}
%  This meta-paper describes the style to be used in articles and short papers
%  for SBC conferences. For papers in English, you should add just an abstract
%  while for the papers in Portuguese, we also ask for an abstract in
%  Portuguese (``resumo''). In both cases, abstracts should not have more than
%  10 lines and must be in the first page of the paper.
%\end{abstract}
%
%\begin{resumo}
%  Este meta-artigo descreve o estilo a ser usado na confecção de artigos e
%  resumos de artigos para publicação nos anais das conferências organizadas
%  pela SBC. É solicitada a escrita de resumo e abstract apenas para os artigos
%  escritos em português. Artigos em inglês deverão apresentar apenas abstract.
%  Nos dois casos, o autor deve tomar cuidado para que o resumo (e o abstract)
%  não ultrapassem 10 linhas cada, sendo que ambos devem estar na primeira
%  página do artigo.
%\end{resumo}

\section{Introdução}

	Este trabalho apresentara resultados da aplicação de métodos de aprendizado de maquinas na predição de movimentos humanos. Foram implementados os seguintes métodos de aprendizado: knn, perceptron e svm. 
	
	Os dados de entrada foram obtidos de \cite{Dua:2017} e são dados de acelerômetro e giroscópio de smartphone junto ao corpo em atividades de levantar, sentar, deitar, caminhar, subir escadas e descer escadas. Os testes foram aplicados em 30 voluntários e os dados foram coletados através dos sensores do smartphone, acelerômetro de 3 eixos e giroscópio de 3 eixos.
	
	Os dados brutos obtidos dos sensores foram pré-processados com a aplicação de filtros para remoção de ruídos, amostrados em intervalos constantes de tempo e normalizados, obtendo um total de 561 características.

	Cada conjunto de dados que representam os movimentos receberam uma \emph{label}, o conjunto de características e o conjunto de \emph{label} são entradas para os algoritmos de classificação.
	
	Como forma de validação do processo de aprendizado, o conjunto de dados de entrada foi dividido em dois subconjuntos denominados conjunto de treino e conjunto de testes. O conjunto de treino é utilizado no processo de treinamento, após o processo de treinamento o conjunto de testes é classificado e os resultados são comparados com as resultados esperados.
	
	Ainda, os algorítimos foram alterados através da redução de dimensionalidade com o uso do método PCA.
	 
	Normalização

	Seleção de características

	Validação

	Como métrica para a determinação da qualidade do processo de aprendizados foram utilizadas a acurácia da predição e utilização do \emph{kappa score} para 

\section{Revisão}

	Esta seção apresenta os métodos aplicados para a classificação dos movimentos.

\subsection{Perceptron}

	\url{http://ml.informatik.uni-freiburg.de/former/_media/documents/teaching/ss09/ml/perceptrons.pdf}

\subsection{KNN}

	\url{http://chem-eng.utoronto.ca/~datamining/Presentations/KNN.pdf}\\
	\url{https://www.ethz.ch/content/dam/ethz/special-interest/gess/computational-social-science-dam/documents/education/Spring2015/datascience/K-Nearest-Neighbour-Classifier.pdf}

\subsection{PCA}

	\url{http://www.iro.umontreal.ca/~pift6080/H09/documents/papers/pca_tutorial.pdf}\\
	\url{http://www.stat.cmu.edu/~cshalizi/350/lectures/10/lecture-10.pdf}

\subsection{SVM}

	\url{http://deeplearning.net/wp-content/uploads/2013/03/dlsvm.pdf}

\subsection{Testes Estatisticos}
	...
	\subsubsection{Validação Cruzada}
	
		\url{https://www.analyticsvidhya.com/blog/2018/05/improve-model-performance-cross-validation-in-python-r/}\\
		
	\subsubsection{Matriz de Confusão}
	
		\url{http://www2.cs.uregina.ca/~hamilton/courses/831/notes/confusion_matrix/confusion_matrix.html}\\
		
	\subsubsection{Kappa Score}
	
		\url{https://www.ncbi.nlm.nih.gov/pmc/articles/PMC3900052/}\\
		\url{http://www.pmean.com/definitions/kappa.htm}\\
		\url{http://www.statisticshowto.com/cohens-kappa-statistic/}\\
		\url{https://academic.oup.com/ptj/article/85/3/257/2805022}
		
\section{Resultados}

	\begin{figure}[ht]
	\centering
	\includegraphics[width=.5\textwidth]{knn_pca_quadratic.png}
	\caption{A typical figure}
	\label{fig:knnpcaquadratic}
	\end{figure}

\begin{table}[!h]
	\begin{tabular}{l|ccc}
		Método &  Acurácia& Kappa Score& \\ \cline{1-4} \cline{1-4}
		KNN &  2&  3 & \\
		Perceptron &  &  &  \\
		SVM &  &  & \\
		KNN + PCA & & & 
	\end{tabular}
	\label{table:resultados}
\end{table}

\section{Conclusões}
	...
%\section{General Information}
%
%All full papers and posters (short papers) submitted to some SBC conference,
%including any supporting documents, should be written in English or in
%Portuguese. The format paper should be A4 with single column, 3.5 cm for upper
%margin, 2.5 cm for bottom margin and 3.0 cm for lateral margins, without
%headers or footers. The main font must be Times, 12 point nominal size, with 6
%points of space before each paragraph. Page numbers must be suppressed.
%
%Full papers must respect the page limits defined by the conference.
%Conferences that publish just abstracts ask for \textbf{one}-page texts.
%
%\section{First Page} \label{sec:firstpage}
%
%The first page must display the paper title, the name and address of the
%authors, the abstract in English and ``resumo'' in Portuguese (``resumos'' are
%required only for papers written in Portuguese). The title must be centered
%over the whole page, in 16 point boldface font and with 12 points of space
%before itself. Author names must be centered in 12 point font, bold, all of
%them disposed in the same line, separated by commas and with 12 points of
%space after the title. Addresses must be centered in 12 point font, also with
%12 points of space after the authors' names. E-mail addresses should be
%written using font Courier New, 10 point nominal size, with 6 points of space
%before and 6 points of space after.
%
%The abstract and ``resumo'' (if is the case) must be in 12 point Times font,
%indented 0.8cm on both sides. The word \textbf{Abstract} and \textbf{Resumo},
%should be written in boldface and must precede the text.
%
%\section{CD-ROMs and Printed Proceedings}
%
%In some conferences, the papers are published on CD-ROM while only the
%abstract is published in the printed Proceedings. In this case, authors are
%invited to prepare two final versions of the paper. One, complete, to be
%published on the CD and the other, containing only the first page, with
%abstract and ``resumo'' (for papers in Portuguese).
%
%\section{Sections and Paragraphs}
%
%Section titles must be in boldface, 13pt, flush left. There should be an extra
%12 pt of space before each title. Section numbering is optional. The first
%paragraph of each section should not be indented, while the first lines of
%subsequent paragraphs should be indented by 1.27 cm.
%
%\subsection{Subsections}
%
%The subsection titles must be in boldface, 12pt, flush left.
%
%\section{Figures and Captions}\label{sec:figs}
%
%
%Figure and table captions should be centered if less than one line
%(Figure~\ref{fig:exampleFig1}), otherwise justified and indented by 0.8cm on
%both margins, as shown in Figure~\ref{fig:exampleFig2}. The caption font must
%be Helvetica, 10 point, boldface, with 6 points of space before and after each
%caption.
%
%\begin{figure}[ht]
%\centering
%\includegraphics[width=.5\textwidth]{fig1.jpg}
%\caption{A typical figure}
%\label{fig:exampleFig1}
%\end{figure}
%
%\begin{figure}[ht]
%\centering
%\includegraphics[width=.3\textwidth]{fig2.jpg}
%\caption{This figure is an example of a figure caption taking more than one
%  line and justified considering margins mentioned in Section~\ref{sec:figs}.}
%\label{fig:exampleFig2}
%\end{figure}
%
%In tables, try to avoid the use of colored or shaded backgrounds, and avoid
%thick, doubled, or unnecessary framing lines. When reporting empirical data,
%do not use more decimal digits than warranted by their precision and
%reproducibility. Table caption must be placed before the table (see Table 1)
%and the font used must also be Helvetica, 10 point, boldface, with 6 points of
%space before and after each caption.
%
%\begin{table}[ht]
%\centering
%\caption{Variables to be considered on the evaluation of interaction
%  techniques}
%\label{tab:exTable1}
%\smallskip
%\begin{tabular}{|l|c|c|}
%\hline
%& Value 1 & Value 2\\[0.5ex]
%\hline
%&&\\[-2ex]
%Case 1 & 1.0 $\pm$ 0.1 & 1.75$\times$10$^{-5}$ $\pm$ 5$\times$10$^{-7}$\\[0.5ex]
%\hline
%&&\\[-2ex]
%Case 2 & 0.003(1) & 100.0\\[0.5ex]
%\hline
%\end{tabular}
%\end{table}
%
%\section{Images}
%
%All images and illustrations should be in black-and-white, or gray tones,
%excepting for the papers that will be electronically available (on CD-ROMs,
%internet, etc.). The image resolution on paper should be about 600 dpi for
%black-and-white images, and 150-300 dpi for grayscale images.  Do not include
%images with excessive resolution, as they may take hours to print, without any
%visible difference in the result. 
%
%\section{References}
%
%Bibliographic references must be unambiguous and uniform.  We recommend giving
%the author names references in brackets, e.g. \cite{knuth:84},
%\cite{boulic:91}, and \cite{smith:99}.
%
%The references must be listed using 12 point font size, with 6 points of space
%before each reference. The first line of each reference should not be
%indented, while the subsequent should be indented by 0.5 cm.

\bibliographystyle{sbc}
\bibliography{sbc-template}

\end{document}
